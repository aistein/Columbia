\documentclass[10pt, letterpaper, twosided]{article}

\usepackage{fancyhdr}
\usepackage{enumerate}
\usepackage{enumitem}
\usepackage{graphicx}
\usepackage{fullpage}
\usepackage{amsmath}
\usepackage{gensymb}
\usepackage{amssymb}
\usepackage{amsthm}
\usepackage{float}
\usepackage{tikz}
\usepackage{pgf}
\usepackage{cancel}

\usepackage{algorithm}
\usepackage{algpseudocode}

\usetikzlibrary{positioning}
\usetikzlibrary{decorations.markings}
\usetikzlibrary{angles,quotes}
\usetikzlibrary{shapes,snakes}


\usepackage{datetime} % \today at \currenttime


\graphicspath{ {Images/} }


\setcounter{section}{0}

\fancyhead{}
\fancyfoot{}

\pagestyle{fancy} % Calling the package

%\fancyhead[CO,CE]{CHM 469 EXAM I}
\fancyfoot[C]{\thepage}
\fancyfoot[R] {Alexander Stein}
\fancyfoot[L] {Algorithms -- HW6}
%\fancyhead[C] {REVISED COPY, QUESTION 5: MC CODEUP}

\renewcommand{\headrulewidth}{0pt}
\renewcommand{\footrulewidth}{0pt}

\graphicspath{ {Images/} }

\newcommand{\veca}{\mathbf{a}}
\newcommand{\vecF}{\mathbf{F}}
\newcommand{\vecr}{\mathbf{r}}
\newcommand{\vecx}{\mathbf{x}}
\newcommand{\vecp}{\mathbf{p}}
\newcommand{\vecv}{\mathbf{v}}
\newcommand{\vecq}{\mathbf{q}}
\newcommand{\vecn}{\mathbf{n}}
\newcommand{\vecf}{\mathbf{f}}
\newcommand{\vecy}{\mathbf{y}}
\newcommand{\vecz}{\mathbf{z}}
\newcommand{\vecL}{\mathbf{L}}
\newcommand{\vecA}{\mathbf{A}}
\newcommand{\vecE}{\mathbf{E}}
\newcommand{\vecH}{\mathbf{H}}
\newcommand{\vecR}{\mathbf{R}}
\newcommand{\vecg}{\mathbf{g}}
\newcommand{\vecJ}{\mathbf{J}}
\newcommand{\veck}{\mathbf{k}}
\newcommand{\vecV}{\mathbf{V}}
\newcommand{\vecN}{\mathbf{N}}
\newcommand{\vecT}{\mathbf{T}}
\newcommand{\vectau}{\boldsymbol{\tau}}
\newcommand{\vecomega}{\boldsymbol{\omega}}
\newcommand{\vecmu}{\boldsymbol{\mu}}
\newcommand{\hcal}{\mathcal{H}}
\newcommand{\dif}{\text{d}}
\newcommand{\vecs}{\mathbf{s}}

%% for section styles
\usepackage{titlesec}

\titleformat*{\section}{\small\bfseries}
\titleformat*{\subsection}{\Large\bfseries}
\titleformat*{\subsubsection}{\large\bfseries}
\titleformat*{\paragraph}{\large\bfseries}
\titleformat*{\subparagraph}{\large\bfseries}

\renewcommand{\thesection}{\Roman{section}} 
%% 

%\usepackage[top=1.875in,bottom=1.875in,left=1.875in,right=1.875in,asymmetric]{geometry}

% LaTeX's margins are, by default, 1.5 inches wide on 12pt documents, 1.75 inches wide on 11pt documents, and 1.875 inches wide on 10pt documents. This is the standard for book margins.

\begin{document}

\thispagestyle{empty}

\noindent \textbf{CSOR 4231 - Introduction to Algorithms HW \#6}

\noindent Alexander Stein (\emph{as5281@columbia.edu})

\noindent due 30 April 2018 at 8:00 PM\\

\noindent \emph{Collaborators: Matthew Carbone, Atishay Seghal}\\

\emph{\textbf{Problem 1 Statement.}} A large store has $m$ customers and
$n$ products and maintains an $m \times n$ matrix $A$ such that
$A_{ij} = 1$ if customer$i$ has purchased product $j$; otherwise,  $A_{ij} = 0$.
Two customers are called $orthogonal$ if they did not purchase any products in common.
Your task is to help the store determine a maximum subset of orthogonal customers.
Give an efficient algorithm for this problem or state its decision version and prove it is $NP$-complete.\\

\emph{\textbf{Problem 1 Solution.}} We posit that this problem is $NP$-complete.  Its decision version 
can be stated as follows:
$$ \text{CUST-ORTH}(D) : \text{find a set of orthogonal customers of size} \geq k.$$
Our proof requires two steps: (1) show CUST-ORTH(D) $\in$ NP; (2) show that
 IS(D) $\leq_{p}$ CUST-ORTH(D).\\
 
\textit{Step 1:}
Let us have the certification algorithm CUST-ORTH-CERT(A, S, k), where A 
is the original matrix, the certificate S is a collection of customers s.t. 
$S_{i} \bot S_{j}, \forall i,j \in m \times n$, and k is the minimum number of such sets required.
\begin{algorithm}
\caption{certify that the given certificate solves CUST-ORTH(D)}\label{ram}
\begin{algorithmic}[1]
\Procedure{CUST-ORTH-CERT}{$A,S,k$}
   \If{$size(S)\leq k$}
      \State \textbf{return} NO
   \EndIf
   \For{$i=1$ to $n$}
      \For{$j=1$ to $m$}
         \If{$j \neq i$ and $S_j \cancel{\bot} S_i$}
            \State \textbf{return} NO
         \EndIf
      \EndFor
   \EndFor
   \State \textbf{return} YES
\EndProcedure
\end{algorithmic}
\end{algorithm}\\
It is trivial to see that this algorithm runs in $O(mn^2)$ time, which is polynomial in the size of the input. 
Thus, $CUST-ORTH(D) \in NP$.\\

\textit{Step 2:}
It is known that the independent-set decision problem (IS(D)) is $NP$-complete. Input to IS(D) is the graph $G(V,E)$,
and the value $k$. Broadly, we'll map edges to products and vertices to customers.
We define our reduction procedure, $R$, from IS(D) to CUST-ORTH(D) as follows:
\begin{itemize}
\item Let $m=|E|$ and $n=|V|$; our $A$ will be size $n \times m$ and initially zero
\item For each edge $e \in E$, give a numeric label to that edge, this will be its column-index $j$ in $A$
\item For each vertex $u \in V$, give a numeric label to that vertex, this will be its row-index $i$ in $A$
\item For each edge $e=(u,v) \in E$, $A_{ue}=1$ and $A_{ve}=1$, where $e,u,v$ are indices as specified above
\item the $k$ value remains the same for both problems
\end{itemize}
Clearly this takes only polynomial time to translate the inputs.  Now it remains to show that - under this translation - 
there is an independent-set of size $\geq k$ for IS(D) $\iff$ there is an orthogonal set of customers of size $\geq k$ 
for CUST-ORTH(D).\\

$\Longrightarrow$ Suppose there is an independent-set of size $k$ for IS(D).  For all pairs of vertices selected from the generated independent set, say $u$ and $v$, we have that $e=(u,v) \cancel{\in} E$.  Thus in $A$, the entries 
$A_{ue}=0$ and $A_{ve}=0$ will be created.  Since the column's of $A$ represent edges, 
and one edge connects two vertices, either both $A_{ae}=1$ and $A_{be}=1$ or both are 0. There 
cannot be some $A_{ij}=1$ and $A_{kj}=1$ if there is no edge (i,j); And there are no edges (i,j) 
between any vertices i and j of the independent-set.  So, under $R$, there will be at least $k$ row vectors which are all mutually orthogonal.\\

$\Longleftarrow$ Suppose there is a set of $k$ orthogonal customers for CUST-ORTH(D).  As described above, there can
only be two 1's in the same column of $A$ if there was originally an edge in G connecting the two vertices corresponding to
these customers.  For every pair $a$ and $b$ of customers in the orthogonal set, $A_{aj}=1 \implies A_{bj}=0$ (and vice. versa) for $j=1,2,...,n$, .  This implies that there could not have been any edges connecting the corresponding vertices of $a$ and $b$ in $G$.  Thus, the set of $k$ orthogonal customers correspond exactly to $k$ vertices in $G$ for which no two are connected by an edge - an independent set.\\

Now we may conclude that CUST-ORTH(D) $\in NPC$. $\blacksquare$\\

\newpage

\emph{\textbf{Probelm 2 Statement.}} Suppose you had a polynomial-time algorithm that, on an input graph, answers
\emph{\textbf{yes}} if and only if the graph has a Hamiltonian cycle.  Show how, on an input graph $G=(V,E)$, you can return 
in polynomial time
\begin{itemize}
\item a Hamiltonian cycle in G, if one exists,
\item \emph{\textbf{no}}, if $G$ does not have a Hamiltonian cycle.
\end{itemize}

\emph{\textbf{Problem 2 Solution.}} The general idea here is to examine one edge of G at a time.  We look at the graph 
$G' = (V, E-\{e\})$  and if HAS-HAM($G'$) = \emph{\textbf{yes}}, then we can throw that edge away because $G'$ still has some Hamiltonian cycle. If it returns \emph{\textbf{no}}, then we conclude that the edge we just removed must have been in every remaining Hamiltonian cycle - put it back in the graph, but mark it as visited so we do not repeat this check.

\begin{algorithm}
\caption{use the HAS-HAM(G) "magic" algorithm to return a Hamiltonian cycle efficiently}\label{ram}
\begin{algorithmic}[1]
\Procedure{GET-HAM}{$G=(V,E)$}
   \If{HAS-HAM(G) = \emph{\textbf{no}}}
      \State \textbf{return} \emph{\textbf{no}}
   \EndIf
   \State Let $crit$ be a bit-vector of length $|E|$ initialized to zeros
   \While{$|E| > |V|$}
      \State Let $e$ be some edge s.t. $e \in E$ and $crit[e]=0$
      \If{HAS-HAM($G'=(V,E - \{e\})$) = \emph{\textbf{yes}}}
         \State remove $e$ from $E$
      \Else
         \State $crit[e] = 1$
      \EndIf
   \EndWhile
   \State \textbf{return} E
\EndProcedure
\end{algorithmic}
\end{algorithm}

First notice that we may terminate when only $|V|$ edges remain, because a simple cycle has exactly $|V|$ edges.  Next,
note that the $crit$ indicator for an edge is only set to 1 if that edge must be in a Hamiltonian cycle.  Upon termination, the 
E set returned will contain only those edges that are in a Hamiltonian cycle, if one exists.

If we let the running of time for HAS-HAM(G) be indicated by H(t), which we know to be polynomial, then the running time
of GET-HAM(G) is $O(mH(t))$, which is also polynomial by composition.\\

\newpage

\emph{\textbf{Problem 3 Statement.}} There is a set of ground elements $E={e_1,e_2,...,e_n}$ and a collection of $m$ subsets $S_1, S_2,...,S_m$ of the ground elements (that is, $S_i \subseteq E \text{ for } 1 \leq i \leq m$).  The goal is
to select a minimum cardinality set $A$ of ground elements such that $A$ contains at least one element from each 
subset $S_i$.  State the decision version of this problem and prove that it is $NP$-complete.\\

\emph{\textbf{Problem 3 Solution.}} Its decision version can be stated as follows:
$$ \text{SUBSET-COVER(D)} : \text{find a subset $A$ as stated above of size} \leq k$$
Our proof requires two steps: (1) show SUBSET-COVER(D) $\in$ NP; (2) show that
VC(D) $\leq_{p}$ CUST-ORTH(D).\\

\textit{Step 1:}
Let us have the certification algorithm SUBSET-COVER-CERT(E, S, A, k), where E 
is the original ground-element-set, S the original collection of subsets, the certificate is A $\subseteq$ E, 
and k is the maximum size of A allowed.
\begin{algorithm}
\caption{certify that the given certificate solves SUBSET-COVER(D)}\label{ram}
\begin{algorithmic}[1]
\Procedure{SUBSET-COVER-CERT}{$E, S, A, k$}
   \If{$size(A)\geq k$}
      \State \textbf{return} NO
   \EndIf
   \State Let $X$ be an array of $|S|$ booleans, false-initialized
   \For{$i=1$ to $|S|$}
      \For{$j=1$ to $|A|$}
         \If{$A[j] \in S_i$}
            \State $X[i] = true$
         \EndIf
      \EndFor
   \EndFor
   \If{any element of X = false}
      \State \emph{\textbf{return}} NO
   \EndIf
   \State \textbf{return} YES
\EndProcedure
\end{algorithmic}
\end{algorithm}\\
Note that $m$ is the number of sets in $S$, and there are at most $n$ of them.  The size of each set is obviously at most
$n$.  So the for-loop on line 4 executes $n$ times.  The inner for-loop on line 5 executes at most $k$ times (which is at
most $n$).  The check for membership on line 6 is also at most $n$.  All other work in this algorithm is dominated by lines 4-7, so the overall running time is $O(n^3)$, which is polynomial.  Thus:  SUBSET-COVER(D) $\in$ NP.\\

\textit{Step 2:}
It is known that the vertex-cover decision problem (VC(D)) is $NP$-complete. Input to VC(D) is the graph $G(V,E)$,
and the value $k$. We define our simple reduction procedure, $R$, from VC(D) to SUBSET-COVER(D) as follows:
\begin{itemize}
\item the vertices of $G$ will be the elements of $E$
\item For each edge $e=(u,v) \in G.E$, create a new subset $S_i = \{u, v\}$ and add it to $S$. (The value of the index doesn't matter)
\item the $k$ value remains the same for both problems
\end{itemize}
Clearly this takes only polynomial time to translate the inputs.  Now it remains to show that - under this translation - 
there is a vertex-cover of size $\leq k$ for VC(D) $\iff$ there is a subset-cover of of size $\leq k$ 
for SUBSET-COVER(D).\\

$\Longrightarrow$ Suppose there is a vertex-cover of size $k$ for VC(D).  Under $R$, every set $S_i \in S$ we create
will have at least one element $e_v$ corresponding to $v$ from that vertex-cover, because there is one set in $S$ for every edge in $G.E$, and the vertex-cover will "touch" every edge at least once (by definition).   Thus, the set $A$ of only those elements corresponding to vertices of the vertex-cover constitutes a subset-cover of size $k$ for SUBSET-COVER(D).\\

$\Longleftarrow$ Suppose there is a subset-cover $A$ of size $k$ for SUBSET-COVER(D).  Every set $S_i \in S$ must 
correspond to an edge in $G.E$.  Every element $e \in A$ must be correspond to a vertex in $G.V$.  So, if $A$ contains at 
least one element from every set, then the vertices corresponding to those elements must "touch" every edge in $G.E$ 
at least once.  Thus, $G$ must have had a vertex-cover of size $k$.\\

Now we may conclude that SUBSET-COVER(D) $\in NPC$. $\blacksquare$\\

\newpage

\emph{\textbf{Problem 4 Statement.}} A paper mill manufactures rolls of paper of standard width 3 meters.  Customers
watn to buy paper rolls of shorter width, and the mill has to cut such rolls from the 3m rolls.  For example, one 3m roll
can be cut into 2 rolls of width 93cm and one roll of width 108 cm; the remaining 6cm goes to waste.  The mill receives 
an order of
\begin{itemize}
\item 97 rolls of width 135 cm
\item 610 rolls of width 108 cm
\item 395 rolls of width 93 cm
\item 211 rolls of width 42 cm
\end{itemize}
Form a linear program to compute the smallest number of 3m rolls that have to be cut to satisfy this order, and explain
how they should be cut.\\

\emph{\textbf{Problem 4 Solution.}} Let us define some terms for clarity:
\begin{itemize}
\item Let $c_0 = 135$, $c_1 = 108$, $c_2 = 93$, and $c_3 = 42$.  These are our \emph{\textbf{cut-types}}.
\item Let $x_i$ represent the total number of 3m rolls which we will cut by cut-method $i$ (defined later).
\item Let $A$ be our coefficient matrix, where $a_{ij} = \text{\# of cuts of type } j \text{ for cut-method } i$ (for a single roll)
\item Let $k_0 = 97$, $k_1 = 610$, $k_2 = 395$, and $k_3 = 211$.  These are the number of resized rolls ordered by each of the 4 customers.
\end{itemize}
Calculating the coefficients of $A$ is a straightforward (albeit tedious) process. What we did to achieve this was as follows:
\begin{itemize}
\item List all the possible combinations of each of the cut-types.  Since there are 4 types, there are $2^4 - 1 = 15$ situations to examine.
\item Each such combination will be of the bounded form $a_0c_0 + a_1c_1 + a_2c_2 + a_3c_3 \leq 300$.  
\item We are considering unique types of combinations or \emph{\textbf{cut-methods}}, i.e. one such method says "we are cutting this 300cm roll into $a_0$ 135-cm pieces, and $a_3$ 42-cm pieces.  As they are unique, writing out a binary-counting-decimal matrix was helpful in keeping track of what combinations were left to be calculated.
\item We solved for the $a_i$ of each cut-method by minimizing the wasted material (300 - sum), and enforcing that each coefficient take on an integer value $>$ 0.  It had to be greater than zero because otherwise it devolved into a different cut-method.  
\item In certain cases, there were more than one set of non-zero integer solutions.  In the case that only one a-value changed from set to set, we just took the set with the largest varying value.  However in the case where multiple a-values change from set to set, we added a new row to $A$ with those constants, because such a set could potentially be used for a more optimal solution during SIMPLEX.
\end{itemize}
Here is the matrix we calculated:
\[
\begin{bmatrix}
	2 & 0 & 0 & 0 \\
	0 & 2 & 0 & 0 \\
	0 & 0 & 3 & 0 \\
	0 & 0 & 0 & 7 \\
	1 & 0 & 1 & 0 \\
	0 & 1 & 2 & 0 \\
	1 & 1 & 0 & 0 \\
	1 & 0 & 0 & 3 \\
	0 & 1 & 0 & 4 \\
	0 & 2 & 0 & 2 \\
	0 & 0 & 1 & 4 \\
	0 & 0 & 2 & 2 \\
	1 & 1 & 0 & 1 \\
	1 & 0 & 1 & 1 \\
	0 & 1 & 1 & 2
\end{bmatrix}
\]\\

Note that there are 15 rows, thus we will have 15 variables $X_i$.  For example, $X_{14}$ represents the number of rolls
on which we will make $1 \times 108$ cm cut, $1 \times 93$ cm cut, and $2 \times 42$ cm cuts.
Now we may finally formulate our Linear Program:\\

Objective: $$\text{Minimize } \sum\limits_{j=1}^n x_j$$\\

Subject to:\\
\begin{enumerate}[label=(\roman*)]
\item $$A^T\vec{x} \geq \vec{k}$$
\item $$\vec{x} \geq \vec{0}$$
\end{enumerate}
The first constraint just says "for each cut type $c_i$, enforce that we create at least $k_i$ such cuts." Note that by the first constraint we are allowing for the possibility of creating more cuts of a particular type than required.  This is necessary because requiring exact equivalence will very likely cause the LP to return "no solution."  

\newpage

\emph{\textbf{Problem 5 Statement.}} Formulate linear or integer programs for the following optimization problems.
(Full-credit will be given to LP solutions, when they are possible.)
\begin{enumerate}[label=(\alph*)]
\item Min-cost flow:  Given a flow network with capacities $c_e$ and costs $a_e$ on every edge $e$, and supplies $s_i$
on every vertex $i$, find a feasible flow $f: E \rightarrow R_+$ -- that is, a flow satisfying edge capacity constraints and node supplies -- that minimizes total cost of the flow.
\item The assignment problem:  There are $n$ persons and $n$ objects that have to be matched on a one-to-one basis.
There's a given set $A$ of ordered pairs $(i,j)$, where a pair $(i,j)$ indicates that person $i$ can be matched with object
$j$.  For every pair $(i,j) \in A$, there's a value $a_{ij}$ for matching person $i$ with object $j$.  Our goal is to assign
persons to objects so as to maximize the total value of the assignment.
\item Uncapacitated facility location:  There is a sret $F$ of $m$ facilities and a set $D$ of $n$ clients.  For each facility
$i \in F$ and each client $j \in D$, there is a cost $c_{ij}$ of assigning client $j$ to facility $i$.  Further, there is a one-time cost $f_i$ associated with opening and operating facility $i$.  Find a subset $F'$ of facilities to open that minimizes the total cost of (i) operating the facilities in $F'$ and (ii) assigning every client $j$ to one of the facilities in $F'$.
\end{enumerate}

\emph{\textbf{Problem 5 Solution.}}
\begin{enumerate}[label=(\alph*)]
\item \underline{Min-cost flow}: For this problem we can formulate a linear program.  Our variables are the amount of flow 
$f_e$ on each edge.  Let $\vec{x}$ be an $1 \times m, m=|E|$ vector representing the flows.  Similarly, $\vec{a}$ is an 
$1 \times m$ vector representing the edge costs, $\vec{c}$ a $1 \times m$ vector representing the capacities, and
$\vec{s}$ an $1 \times n, n=|V|$ vector representing the supplies of the vertices.\\

Objective: $$\text{Minimize } \vec{a}\vec{x}^T$$\\

Subject To:\\
\begin{enumerate}[label=(\roman*)]
\item $$\sum\limits_{j: e_j \text{ out of } i}x_j  - \sum\limits_{j: e_j \text{ in to } i}x_j = s_i \text{  ,  } i = 1,2,...,n$$
\item $$\sum\limits_{i \in V}\sum\limits_{j : e_j\text{ out of } i}x_j  -  \sum\limits_{i \in V}\sum\limits_{j : e_j\text{ in to } i}x_j = 0\text{  ,  } i = 1,2,...,n$$
\item $$\vec{x} \leq \vec{c}$$
\item $$\vec{x} \geq \vec{0}$$
\end{enumerate}

\item \underline{Assignment Problem}:  This problem too can be formulated as a linear program, but first we must transform
the inputs a little bit - into a graph!  The set $A$ we will move directly to $G.E$.  The persons $i$ and objects $j$ will all be
added to $G.V$ as individual vertices.  Now it should be apparent that we have a bipartite graph, because there will be no
edge between any person nodes, and no edge between any object nodes -- only between person and object.  The weights of each edge will equal $-a_{ij}$, because we wish to maximize "cost" instead of minimize it. Furthermore, since the problem requires an exact one-to-one correspondence, this graph problem is now a "maximum bipartite matching" problem.\\

Unfortunately, we're not done yet!  We will translate the problem into Min-cost-flow, for which we already formulated an LP
in part (a).  To do so we simply (1) add the source and sink nodes $s$ and $t$ to $G.V$; (2) for every person node $i$ add the zero-cost-edge $(s,i)$ to $G.E$, and for every object node $j$ add the zero-cost-edge $(j,t)$ to $G.E$; (3) let the capacity of every edge $c_e$ in the graph be 1; (4) let all supplies $s_i=0$, $s_j=0$ except $s_s=n$ and $s_t=-n$.
When this new input is solved for min-cost-flow, we can take the person-object edges selected by that flow as our maximum value assignment. \\

\item \underline{Uncapacitated facility location}:  For this one we are stuck with an ($NP$-complete) integer program formulation.  We'll use $m \times n$ indicator variables: 
\[
x_{ij} =
\begin{cases}
			1 & \text{ if client $j$ is assigned to facility $i$} \\
			0 & \text{ otherwise}
\end{cases}
\]
If we define cost as "cost of opening facilities" + "cost of assigning customers" then we can define our integer program as follows:\\

Objective: $$\text{Minimize: } \sum\limits_{i=1}^m f_i * \max_{1 \leq j < n} x_{ij} + \sum\limits_{i=1}^m\sum\limits_{j=1}^n x_{ij}*c_{ij}$$\\

Subject To:\\
\begin{enumerate}[label=(\roman*)]
\item $$\sum\limits_{i=1}^m x_{ij} = 1 \text{ for } j=1,2,...,n$$
\item $$x_{ij} \in \{0,1\}$$
\end{enumerate}
Note that the quantity $\max_{1 \leq j < n} x_{ij} \in \{0,1\}$ because though a facility may have multiple customers assigned
to it, it is only every opened once!

\end{enumerate}













\end{document}